\chapter{Аналитический раздел}

В данном разделе будет сформулирована задача коммивояжера, а также будут рассмотрены 2 метода решения этой задачи: муравьиный алгоритм и алгоритм
полного перебора.

\section{Формулировка задачи коммивояжера}

Пусть задан граф $G = (V, E)$, где $V$ --- множество вершин ($|V|=n$), а $E$ --- множество рeбер ($|E|=m$). Каждое ребро ($(i,j)\in E$ имеет длину $c_{ij}$, определяемую матрицей расстояний $C=||c_{ij}||$. Если между вершинами $i$ и $j$ отсутствует ребро, соответствующий элемент матрицы принимается равным бесконечности ($c_{ij} = \infty$)~\cite{com_info}.

Подмножество попарно несмежных ребер графа $G$ называется паросочетанием. Паросочетание считается совершенным, если каждая вершина графа инцидентна ровно одному ребру из этого множества. Совокупность простых попарно непересекающихся циклов, покрывающая все вершины графа $G$, называется 2-фактором. Если 2-фактор состоит из одного цикла, то он называется гамильтоновым циклом~\cite{gamelton}.

Задача заключается в поиске гамильтонова цикла минимальной длины, то есть цикла, который проходит через каждую вершину графа ровно один раз и возвращается в начальную точку.


\subsection{Алгоритм полного перебора}
Алгоритм полного перебора для решения задачи коммивояжера основывается на проверке всех возможных маршрутов в графе для определения минимального. Этот метод заключается в полном переборе всех вариантов обхода городов и выборе маршрута с наименьшей длиной. Однако количество возможных маршрутов быстро растет с увеличением числа городов $n$, поскольку сложность алгоритма составляет $n!$. Несмотря на то, что данный подход гарантирует получение точного решения, его применение становится крайне неэффективным даже при относительно небольшом количестве городов из-за значительных вычислительных затрат.

\subsection{Муравьиный алгоритм}
В основе муравьиного алгоритма лежит идея моделирования поведения колонии муравьев. Каждый муравей определяет свой маршрут на основе оставленных другими муравьями феромонов, а также сам оставляет оставляет феромоны, чтобы последующие муравьи ориентировались по ним. В результате при прохождении каждым муравьем своего маршрута наибольшее число феромонов остается на самом оптимальном пути. Временная сложность алгоритма была оценена как $683 - (42,467 · N) + (1,0696 · N^2)$~\cite{ants}. Однако главный недостаток алгоритма заключается в том, что, по сравнению с алгоритмом полного перебора, он дает приближенное решение задачи, а не точное.

Вероятность перехода муравья $k$ из текущей вершины $i$ в вершину $j$ рассчитывается по формуле:
\begin{equation}
    \label{posib}
    P_{kij} = 
    \begin{cases}
        \frac{\tau_{ij}^a \eta_{ij}^b}{\sum_{q \in J_{ik}} \tau_{iq}^a \eta_{iq}^b}, & \text{если вершина $j$ еще не посещена муравьем $k$,} \\
        0, & \text{иначе,}
    \end{cases}
\end{equation}
где:
\begin{itemize}
    \item[---] $a$ --- параметр влияния феромона;
    \item[---] $b$ --- параметр влияния длины пути;
    \item[---] $\tau_{ij}$ --- количество феромонов на ребре $(i, j)$;
    \item[---] $\eta_{ij}$ --- видимость (величина обратная расстоянию до вершины).
\end{itemize}

По окончанию движения всех муравьев уровень феромонов на ребрах обновляется по формуле:
\begin{equation}
    \label{update_phero_1}
    \tau_{ij}(t+1) = (1-p)\tau_{ij}(t) + \Delta \tau_{ij},
\end{equation}

где $p$ — коэффициент испарения феромона, а $\Delta \tau_{ij}$ определяется как:

\begin{equation}
    \label{update_phero_2}
    \Delta \tau_{ij} = \sum_{k=1}^N \Delta \tau_{ij}^k,
\end{equation}

\begin{equation}
    \label{update_phero_3}
    \Delta \tau_{ij}^k = 
    \begin{cases}
        \frac{Q}{L_k}, & \text{если ребро $(i, j)$ посещено муравьем $k$,} \\
        0, & \text{иначе,}
    \end{cases}
\end{equation}

где $Q$ — параметр, связанный с длиной оптимального пути, а $L_k$ — длина маршрута муравья $k$.

\subsection{Описание алгоритма}
Пошаговое описание муравьиного алгоритма:
\begin{enumerate}
    \item[1)] Муравей исключает из дальнейшего выбора вершины, которые уже были посещены, ссылаясь на список посещенных вершин, хранящийся в его памяти (список запретов $J_{ik}$).
    \item[2)] Муравей оценивает привлекательность вершин, основываясь на их видимости, которая обратно пропорциональна расстоянию между ними.
    \item[3)] Муравей ощущает уровень феромонов на ребрах графа, что помогает ему определять предпочтительность маршрута.
    \item[4)] После прохождения ребра $(i, j)$ муравей оставляет на нем феромоны, количество которых зависит от длины маршрута $L_k$, пройденного муравьем, и параметра $Q$.
\end{enumerate}

\textbf{ВЫВОД}

В данном разделе была представлена формулировка задача коммивояжера, а также рассмотрены 2 метода ее решения: муравьиный алгоритм и алгоритм полного перебора
