\begin{center}
    \textbf{ВВЕДЕНИЕ}
\end{center}
\addcontentsline{toc}{chapter}{ВВЕДЕНИЕ}

Задача коммивояжера --- одна из самых известных и старейших задач комбинаторной оптимизации. Ее корни уходят в 1831 год, когда в Германии была опубликована книга под названием "Кто такой коммивояжер и что он должен делать для процветания своего предприятия". В книге содержалась рекомендация: "Следует стремиться посетить как можно больше торговых точек, избегая повторного посещения". Это можно считать первым формулированием задачи коммивояжера~\cite{com_info}.

\textbf{Цель лабораторной работы} --- рассмотрение алгоритмов решения задачи коммивояжера.

Для достижения поставленной цели необходимо выполнить следующие задачи:
\begin{itemize}
    \item[---] сформулировать задачу коммивояжера;
    \item[---] рассмотреть алгоритмы решения: полным перебором, с использованием муравьиного алгоритма;
    \item[---] реализовать данные алгоритмы;
    \item[---] провести сравнительный анализ времени работы алгоритмов;
    \item[---] выполнить параметризацию для муравьиного алгоритма.
\end{itemize}