\begin{center}
    \textbf{ЗАКЛЮЧЕНИЕ}
\end{center}
\addcontentsline{toc}{chapter}{ЗАКЛЮЧЕНИЕ}

В процессе работы были исследованы временные и алгоритмические сложности муравьиного алгоритма и метода полного перебора. Также были проведены замеры времени выполнения и параметризация муравьиного алгоритма, что дало возможность определить оптимальные параметры для набора данных, представленного в приложении А. Наилучшие параметры были установлены на уровне:  $\alpha = 0.1$, $\rho=0.5$ и количество дней равным 50.


В ходе выполнения данной лабораторной работы были решены следующие задачи:
\begin{itemize} 
    \item[---] сформулирована задача коммивояжера; 
    \item[---] рассмотреть алгоритмы решения: полным перебором, с использованием муравьиного алгоритма; 
    \item[---] реализовать данные алгоритмы; 
    \item[---] провести сравнительный анализ времени работы алгоритмов; 
    \item[---] выполнить параметризацию для муравьиного алгоритма. 
\end{itemize}