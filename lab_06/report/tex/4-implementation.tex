\chapter{Технологический раздел}

В данном разделе будут приведены средства реализации, листинги кода.

\section{Средства реализации}
Для реализации данной лабораторной работы выбран язык программирования \textit{C++}~\cite{C++}. Выбор обусловлен скоростью выполнения и наличием множества библиотек, упрощающих разработку.

\section{Реализация алгоритмов}
В листингах~\ref{lst:default} ---~\ref{lst:last} представлены реализации алгоритмов.

\begin{center}
\captionsetup{justification=raggedright,singlelinecheck=off}
\begin{lstlisting}[language=C++, frame=single, numbers=left, label=lst:default, caption=Реализация алгоритма полного перебора]
std::vector<int> bruteForceTSP(const std::vector<std::vector<int>>& graph) {
    int n = graph.size();
    std::vector<int> vertices(n);
    for (int i = 0; i < n; ++i) vertices[i] = i;

    std::vector<int> bestCycle;
    double minDistance = numeric_limits<double>::max();

    do {
        double currentDistance = calculateDistance(vertices, graph);
        if (currentDistance < minDistance) {
            minDistance = currentDistance;
            bestCycle = vertices;
        }
    } while (next_permutation(vertices.begin() + 1, vertices.end()));

    return bestCycle;
}
\end{lstlisting}
\end{center}


\begin{center}
\captionsetup{justification=raggedright,singlelinecheck=off}
\begin{lstlisting}[language=C++, frame=single, numbers=left, label=lst:vinograd,caption=Реализация функции обновления ферамонов]
void updatePheromones(std::vector<std::vector<double>>& pheromones, const std::vector<int>& bestCycle, double bestDistance) {
    int n = pheromones.size();
    for (int i = 0; i < n; ++i) {
        for (int j = 0; j < n; ++j) {
            pheromones[i][j] *= (1.0 - EVAPORATION_RATE);
        }
    }

    for (size_t i = 0; i < bestCycle.size() - 1; ++i) {
        int u = bestCycle[i];
        int v = bestCycle[i + 1];
        pheromones[u][v] += 1.0 / bestDistance;
        pheromones[v][u] += 1.0 / bestDistance;
    }
}
\end{lstlisting}
\end{center}

\begin{center}
\captionsetup{justification=raggedright,singlelinecheck=off}
\begin{lstlisting}[language=C++, frame=single, numbers=left, label=lst:vinograd_mod, caption=Реализация функции выбора следующей вершины]
int chooseNextCity(int current, const std::vector<bool>& visited, const std::vector<std::vector<double>>& pheromones, const std::vector<std::vector<int>>& graph) {
    int n = graph.size();
    std::vector<double> probabilities(n, 0.0);
    double sum = 0.0;

    for (int next = 0; next < n; ++next) {
        if (!visited[next]) {
            probabilities[next] = pow(pheromones[current][next], ALPHA) * pow(1.0 / graph[current][next], BETA);
            sum += probabilities[next];
        }
    }

    double randomValue = ((double)rand() / RAND_MAX) * sum;
    double cumulative = 0.0;
    for (int next = 0; next < n; ++next) {
        if (!visited[next]) {
            cumulative += probabilities[next];
            if (cumulative >= randomValue) {
                return next;
            }
        }
    }
}
\end{lstlisting}
\end{center}

\begin{center}
\captionsetup{justification=raggedright,singlelinecheck=off}
\begin{lstlisting}[language=C++, frame=single, numbers=left, label=lst:last, caption=Реализация муравьиного алгоритма]
vector<int> antColonyOptimization(const std::vector<std::vector<int>>& graph) {
    size_t n = graph.size();
    std::vector<std::vector<double>> pheromones(n, vector<double>(n, 1.0));
    std::vector<int> bestCycle;
    double bestDistance = numeric_limits<double>::max();

    srand(time(0));

    for (int iter = 0; iter < ITERATIONS; ++iter) {
        for (size_t start = 0; start < n; ++start) {
            std::vector<bool> visited(n, false);
            std::vector<int> cycle;
            int current = start;
            cycle.push_back(current);
            visited[current] = true;

            while (cycle.size() < n) {
                int next = chooseNextCity(current, visited, pheromones, graph);
                if (next == -1) break;
                cycle.push_back(next);
                visited[next] = true;
                current = next;
            }

            if (cycle.size() == n && graph[cycle.back()][cycle[0]] > 0) {
                cycle.push_back(cycle[0]);
                double distance = calculateDistance(cycle, graph);
                if (distance < bestDistance) {
                    bestDistance = distance;
                    bestCycle = cycle;
                }
            }
        }

        if (!bestCycle.empty()) {
            updatePheromones(pheromones, bestCycle, bestDistance);
        }
    }

    return bestCycle;
}
\end{lstlisting}
\end{center}


\textbf{ВЫВОД}

В данном разделе были представлены реализации алгоритмов решения задачи коммивояжера: алгоритм полного перебора, муравьиный алгоритм.

\clearpage