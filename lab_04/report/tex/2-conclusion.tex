\begin{center}
    \textbf{ЗАКЛЮЧЕНИЕ}
\end{center}
\addcontentsline{toc}{chapter}{ЗАКЛЮЧЕНИЕ}

Цель работы достигнута: сравнены основные принципы последовательных вычислений с параллельными на основе нативных потоков. Было выявлено, что однопоточная реализация программы работает значительно медленнее по сравнению с многопоточной версией. При фиксированном количестве потоков и увеличении числа обрабатываемых страниц время выполнения однопоточной программы примерно в 8 раз больше, чем у 16-поточной версии. При увеличении количества потоков и фиксированном числе обрабатываемых страниц время выполнения сокращается приблизительно в 2 раза при удвоении числа потоков.

\vspace{5mm}

В ходе выполнения данной лабораторной работы были решены следующие задачи:
\begin{itemize}
    \item[---] описаны входные, выходные данные;
    \item[---] реализованы два алгоритма для загрузки контента из $HTML$---страниц: последовательный и параллельный с использованием нативных потоков;
    \item[---] протестированы разработанные алгоритмы;
    \item[---] выполнено сравнение скорости выполнения программы в зависимости от количества используемых потоков.
\end{itemize}