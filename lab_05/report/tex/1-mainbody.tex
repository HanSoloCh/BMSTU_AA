\begin{center}
    \textbf{ВВЕДЕНИЕ}
\end{center}
\addcontentsline{toc}{chapter}{ВВЕДЕНИЕ}

В современном программировании эффективная обработка данных играет ключевую роль в повышении производительности приложений. Одним из методов оптимизации является конвейерная обработка, которая разделяет процесс обработки данных на отдельные последовательные этапы. Это позволяет каждой стадии работать независимо, что способствует параллельной обработке и более рациональному использованию системных ресурсов.

\textbf{Цель лабораторной работы} --- освоение навыков организации параллельных вычислений на основе принципа конвейерной обработки. Для достижения этой цели необходимо решить следующие задачи:

\begin{itemize}
	\item[---] описать входные, выходные данные программы;
	\item[---] реализовать алгоритм обработки данных с использованием конвейерной обработки;
	\item[---] протестировать разработанный алгоритм по методологии черного ящика;
        \item[---] проанализировать полученную из лог-файлов информацию.
\end{itemize}


\chapter{Входные и выходные данные}

Входные данные: директория со скачанными рецептами для обработки.

Выходные данные: $SQLite$ база данных, содержащая информацию о каждом рецепте ($ID$, номер варианта, $URL$, название, ингредиенты, шаги); лог-файлы, фиксирующие время начала и завершения потоком обработки задачи.

\chapter{Тестирование}
В таблице~\ref{tbl:tests} представлены функциональные тесты для разработанного
программного обеспечения. Все тесты пройдены успешно.

\begin{table}[h!]
    \begin{center}
		\begin{threeparttable}
    \caption{Описание тестовых случаев}
    \captionsetup{justification=raggedright, singlelinecheck=false}
    \label{tbl:tests}
    \begin{tabular}{|c|p{6cm}|p{6cm}|c|}
        \hline
        \textbf{№} & \textbf{Входные данные} & \textbf{Ожидаемый результат} & \textbf{Результат теста} \\
        \hline
        1 & Директория с выгруженными рецептами \texttt{recipes} & База данных $MySQL$  & Пройден \\
        \hline
        2 & Пустая директория & Вывод сообщения об ошибке & Пройден \\
        \hline
        3 & Неверная директория & Вывод сообщения об ошибке & Пройден \\
        \hline
    \end{tabular}
    \end{threeparttable}
    \end{center}
\end{table}



\chapter{Описание исследования}
В ходе исследования требуется сформировать лог обработки задач для наглядной демонстрации работы параллельных потоков по конвейерному принципу. 

\section{Технические характеристики}
Технические характеристики, используемого устройства:
\begin{itemize}
    \item[---] операционная система --- Ubuntu Linux x86\_64~\cite{Ubuntu};
    \item[---] память --- 16 Гб;
    \item[---] процессор --- AMD Ryzen 5 5500U (6x2.10 ГГц)~\cite{AMD}.
\end{itemize}

\section{Полученные результаты}

В таблице~\ref{tbl:log1} приведен лог для 5 задач.

\begin{table}[h]
	\begin{center}
		\begin{threeparttable}
		\captionsetup{justification=raggedright,singlelinecheck=off}
		\caption{Лог выполнения программы.}
		\label{tbl:log1}
                    \begin{tabular}{|l|l|}
                        \hline
                        Время & Событие\\
                        \hline
                         13:02:12.107  & задача 1 начала обрабатываться 1 обработчиком \\
                        \hline
                         13:02:12.107  & задача 1 закончила обрабатываться 1 обработчиком \\
                        \hline
                         13:02:12.107  & задача 2 начала обрабатываться 1 обработчиком \\
                        \hline
                         13:02:12.107  & задача 2 закончила обрабатываться 1 обработчиком \\
                        \hline
                         13:02:12.107  & задача 3 начала обрабатываться 1 обработчиком \\
                        \hline
                         13:02:12.107  & задача 3 закончила обрабатываться 1 обработчиком \\
                        \hline
                         13:02:12.107  & задача 4 начала обрабатываться 1 обработчиком \\
                         \hline
                         13:02:12.107  & задача 4 закончила обрабатываться 1 обработчиком \\
                        \hline
                         13:02:12.107  & задача 5 начала обрабатываться 1 обработчиком \\
                        \hline
                         13:02:12.107  & задача 5 закончила обрабатываться 1 обработчиком \\
                        \hline
                         13:02:12.107  & задача 1 начала обрабатываться 2 обработчиком \\
                                                 \hline

                         13:02:12.109  & задача 1 закончила обрабатываться 2 обработчиком \\
                                                 \hline

                         13:02:12.109  & задача 2 начала обрабатываться 2 обработчиком \\
                                                 \hline

                         13:02:12.110  & задача 2 закончила обрабатываться 2 обработчиком \\
                                                 \hline

                         13:02:12.110  & задача 3 начала обрабатываться 2 обработчиком \\
                                                 \hline

                         13:02:12.111  & задача 3 закончила обрабатываться 2 обработчиком \\
                                                 \hline

                         13:02:12.111  & задача 4 начала обрабатываться 2 обработчиком \\
                                                 \hline

                         13:02:12.112  & задача 4 закончила обрабатываться 2 обработчиком \\
                                                 \hline

                         13:02:12.112  & задача 5 начала обрабатываться 2 обработчиком \\
                                                 \hline

                         13:02:12.113  & задача 5 закончила обрабатываться 2 обработчиком \\
                                                 \hline

                         13:02:12.109  & задача 1 начала обрабатываться 3 обработчиком \\
                                                 \hline

                         13:02:12.109  & задача 1 закончила обрабатываться 3 обработчиком \\
                                                 \hline

                         13:02:12.110  & задача 2 начала обрабатываться 3 обработчиком \\
                                                 \hline

                         13:02:12.110  & задача 2 закончила обрабатываться 3 обработчиком \\
                                                 \hline

                         13:02:12.111  & задача 3 начала обрабатываться 3 обработчиком\\
                                                 \hline

                         13:02:12.111  & задача 3 закончила обрабатываться 3 обработчиком \\
                                                 \hline

                         13:02:12.112  & задача 4 начала обрабатываться 3 обработчиком \\
                                                 \hline

                         13:02:12.112  & задача 4 закончила обрабатываться 3 обработчиком \\
                                                 \hline

                         13:02:12.113  & задача 5 начала обрабатываться 3 обработчиком \\
                                                 \hline

                         13:02:12.113  & задача 5 закончила обрабатываться 3 обработчиком \\
                                                 \hline


                        \hline
                    \end{tabular}
		\end{threeparttable}
    \end{center}
\end{table}


\clearpage
В результате анализа логов было подтверждено, что конвейерная обработка задач выполняется параллельно. Особенно заметно это при рассмотрении работы второго и третьего обработчиков. Закончив обработку текущей задачи, обработчик берет следующую задачу с конвейера. Если конвейер пуст, то обработчик ожидает, пока задача не появится. Такая обработка получается производительнее, чем последовательная обработка.


Среднее время обработки задачи --- $0.54, 1.49, 0,54$ секунд для первого, второго и третьего обработчика соответственно.
Среднее время ожидания задачи в очереди --- $0.44, 1.25, 0.67$ для первой, второй и третьей очереди соответственно.

\clearpage


