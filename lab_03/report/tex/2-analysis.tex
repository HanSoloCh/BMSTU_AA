\chapter{Аналитический раздел}

В данном разделе рассмотрены алгоритмы нахождения заданного значения в множестве.

\section{Описание алгоритмов}

\subsection{Алгоритм линейного поиска}
При использовании алгоритма линейного поиска происходит перебор всех элементов множества~\cite{linear_search}. Поэтому скорость нахождения значения зависит от его расположении в множестве.

\subsection{Алгоритм бинарного поиска}
В алгоритме бинарного поиска перебор всех элементов множества не осуществляется. Для его работы множество должно быть отсортировано. Определяются левая и правая границы поиска, после чего выбирается элемент, находящийся в середине этого диапазона, и он сравнивается с искомым значением. Если искомое значение меньше среднего элемента, то правая граница сдвигается влево, за средний элемент. Если искомое значение больше, то левая граница перемещается вправо. Этот процесс повторяется до тех пор, пока искомое значение не совпадет с центральным элементом или область поиска не сократится до нуля~\cite{binary_search}.

\clearpage
\textbf{ВЫВОД}

В данном разделе были рассмотрены алгоритм линейного поиска и алгоритм бинарного поиска заданного значения в множестве.