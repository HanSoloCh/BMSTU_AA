\begin{center}
    \textbf{ЗАКЛЮЧЕНИЕ}
\end{center}
\addcontentsline{toc}{chapter}{ЗАКЛЮЧЕНИЕ}

В результате исследования было определено, что классический алгоритм умножения матриц проигрывает по времени алгоритму Винограда примерно в 1.2 раза из-за того, что в алгоритме Винограда часть вычислений происходит заранее, а также сокращается часть сложных операций - операций умножения, поэтому предпочтение следует отдавать алгоритму Винограда. Но лучшие показатели по времени выдает оптимизированный алгоритм Винограда -- он примерно в 1.2 раза быстрее алгоритма Винограда на размерах матриц свыше 10 из-за замены операций равно и плюс на операцию плюс-равно, за счет замены операции умножения операцией сдвига, а также за счет предвычислений некоторых слагаемых, что дает проводить часть вычислений быстрее. Поэтому при выборе самого быстрого алгоритма предпочтение стоит отдавать оптимизированному алгоритму Винограда.


В ходе выполнения данной лабораторной работы были решены следующие задачи:
\begin{itemize} 
    \item[---] разобраны алгоритмы умножения матриц: стандартный, Винограда и оптимизированный алгоритм Винограда; 
    \item[---] выполнена оценка трудоемкости алгоритмов; 
    \item[---] реализованы алгоритмы умножения матриц: стандартный, Винограда и оптимизированный алгоритм Винограда; 
    \item[---] выполнены замеры используемого процессорного времени алгоритмами в зависимости от размера входных матриц; 
    \item[---] описаны полученные результаты в отчете. 
\end{itemize}